\documentclass[english]{article}
\newcommand{\G}{\overline{C_{2k-1}}}
\usepackage[latin9]{inputenc}
\usepackage{amsmath}
\usepackage{amssymb,mathabx}
\usepackage{lmodern}
\usepackage{mathtools}
\usepackage[inline]{enumitem}
\usepackage{relsize}
\usepackage{tikz-cd}
%\usepackage{natbib}
%\bibliographystyle{plainnat}
%\setcitestyle{authoryear,open={(},close={)}}
\let\avec=\vec
\renewcommand\vec{\mathbf}
\renewcommand{\d}[1]{\ensuremath{\operatorname{d}\!{#1}}}
\newcommand{\pydx}[2]{\frac{\partial #1}{\partial #2}}
\newcommand{\dydx}[2]{\frac{\d #1}{\d #2}}
\newcommand{\ddx}[1]{\frac{\d{}}{\d{#1}}}
\newcommand{\hk}{\hat{K}}
\newcommand{\hl}{\hat{\lambda}}
\newcommand{\ol}{\overline{\lambda}}
\newcommand{\om}{\overline{\mu}}
\newcommand{\all}{\text{all }}
\newcommand{\valph}{\vec{\alpha}}
\newcommand{\vbet}{\vec{\beta}}
\newcommand{\vT}{\vec{T}}
\newcommand{\vN}{\vec{N}}
\newcommand{\vB}{\vec{B}}
\newcommand{\vX}{\vec{X}}
\newcommand{\vx}{\vec {x}}
\newcommand{\vn}{\vec{n}}
\newcommand{\vxs}{\vec {x}^*}
\newcommand{\vV}{\vec{V}}
\newcommand{\vTa}{\vec{T}_\alpha}
\newcommand{\vNa}{\vec{N}_\alpha}
\newcommand{\vBa}{\vec{B}_\alpha}
\newcommand{\vTb}{\vec{T}_\beta}
\newcommand{\vNb}{\vec{N}_\beta}
\newcommand{\vBb}{\vec{B}_\beta}
\newcommand{\bvT}{\bar{\vT}}
\newcommand{\ka}{\kappa_\alpha}
\newcommand{\ta}{\tau_\alpha}
\newcommand{\kb}{\kappa_\beta}
\newcommand{\tb}{\tau_\beta}
\newcommand{\hth}{\hat{\theta}}
\newcommand{\evat}[3]{\left. #1\right|_{#2}^{#3}}
\newcommand{\restr}[2]{\evat{#1}{#2}{}}
\newcommand{\prompt}[1]{\begin{prompt*}
		#1
\end{prompt*}}
\newcommand{\vy}{\vec{y}}
\DeclareMathOperator{\sech}{sech}
\DeclarePairedDelimiter\abs{\lvert}{\rvert}%
\DeclarePairedDelimiter\norm{\lVert}{\rVert}%
\newcommand{\dis}[1]{\begin{align}
	#1
	\end{align}}
\newcommand{\LL}{\mathcal{L}}
\newcommand{\RR}{\mathbb{R}}
\newcommand{\CC}{\mathbb{C}}
\newcommand{\NN}{\mathbb{N}}
\newcommand{\ZZ}{\mathbb{Z}}
\newcommand{\QQ}{\mathbb{Q}}
\newcommand{\Ss}{\mathcal{S}}
\newcommand{\BB}{\mathcal{B}}
\usepackage{graphicx}
% Swap the definition of \abs* and \norm*, so that \abs
% and \norm resizes the size of the brackets, and the 
% starred version does not.
%\makeatletter
%\let\oldabs\abs
%\def\abs{\@ifstar{\oldabs}{\oldabs*}}
%
%\let\oldnorm\norm
%\def\norm{\@ifstar{\oldnorm}{\oldnorm*}}
%\makeatother
\newenvironment{subproof}[1][\proofname]{%
	\renewcommand{\qedsymbol}{$\blacksquare$}%
	\begin{proof}[#1]%
	}{%
	\end{proof}%
}

\usepackage{centernot}
\usepackage{dirtytalk}
\usepackage{calc}
\newcommand{\prob}[1]{\setcounter{section}{#1-1}\section{}}


\newcommand{\prt}[1]{\setcounter{subsection}{#1-1}\subsection{}}
\newcommand{\pprt}[1]{{\textit{{#1}.)}}\newline}
\renewcommand\thesubsection{\alph{subsection}}
\usepackage[sl,bf,compact]{titlesec}
\titlelabel{\thetitle.)\quad}
\DeclarePairedDelimiter\floor{\lfloor}{\rfloor}
\makeatletter

\newcommand*\pFqskip{8mu}
\catcode`,\active
\newcommand*\pFq{\begingroup
	\catcode`\,\active
	\def ,{\mskip\pFqskip\relax}%
	\dopFq
}
\catcode`\,12
\def\dopFq#1#2#3#4#5{%
	{}_{#1}F_{#2}\biggl(\genfrac..{0pt}{}{#3}{#4}|#5\biggr
	)%
	\endgroup
}
\def\res{\mathop{Res}\limits}
% Symbols \wedge and \vee from mathabx
% \DeclareFontFamily{U}{matha}{\hyphenchar\font45}
% \DeclareFontShape{U}{matha}{m}{n}{
%       <5> <6> <7> <8> <9> <10> gen * matha
%       <10.95> matha10 <12> <14.4> <17.28> <20.74> <24.88> matha12
%       }{}
% \DeclareSymbolFont{matha}{U}{matha}{m}{n}
% \DeclareMathSymbol{\wedge}         {2}{matha}{"5E}
% \DeclareMathSymbol{\vee}           {2}{matha}{"5F}
% \makeatother

%\titlelabel{(\thesubsection)}
%\titlelabel{(\thesubsection)\quad}
\usepackage{listings}
\lstloadlanguages{[5.2]Mathematica}
\usepackage{babel}
\newcommand{\ffac}[2]{{(#1)}^{\underline{#2}}}
\usepackage{color}
\usepackage{amsthm}
\newtheorem{theorem}{Theorem}[section]
\newtheorem*{theorem*}{Theorem}
\newtheorem{conj}[theorem]{Conjecture}
\newtheorem{corollary}[theorem]{Corollary}
\newtheorem{example}[theorem]{Example}
\newtheorem{lemma}[theorem]{Lemma}
\newtheorem*{lemma*}{Lemma}
\newtheorem{problem}[theorem]{Problem}
\newtheorem{proposition}[theorem]{Proposition}
\newtheorem*{proposition*}{Proposition}
\newtheorem*{corollary*}{Corollary}
\newtheorem{fact}[theorem]{Fact}
\newtheorem*{prompt*}{Prompt}
\newtheorem*{claim*}{Claim}
\newtheorem{claim}[theorem]{Claim}
%\newcommand{\claim}[1]{\begin{claim*} #1\end{claim*}}
%organizing theorem environments by style--by the way, should we really have definitions (and notations I guess) in proposition style? it makes SO much of our text italicized, which is weird.
\theoremstyle{remark}
\newtheorem{remark}{Remark}[section]

\theoremstyle{definition}
\newtheorem{definition}[theorem]{Definition}
\newtheorem*{definition*}{Definition}
\newtheorem{notation}[theorem]{Notation}
\newtheorem*{notation*}{Notation}
%FINAL
\newcommand{\due}{9 May 2018} 
\RequirePackage{geometry}
\geometry{margin=.7in}
\usepackage{todonotes}
\title{MATH 8302 Take-Home Final}
\author{David DeMark}
\date{\due}
\usepackage{fancyhdr}
\pagestyle{fancy}
\fancyhf{}
\rhead{David DeMark}
\chead{\due}
\lhead{MATH 8302}
\cfoot{\thepage}
% %%
%%
%%
%DRAFT

%\usepackage[left=1cm,right=4.5cm,top=2cm,bottom=1.5cm,marginparwidth=4cm]{geometry}
%\usepackage{todonotes}
% \title{MATH 8669 Homework 4-DRAFT}
% \usepackage{fancyhdr}
% \pagestyle{fancy}
% \fancyhf{}
% \rhead{David DeMark}
% \lhead{MATH 8669-Homework 4-DRAFT}
% \cfoot{\thepage}

%PROBLEM SPEFICIC

\newcommand{\lint}{\underline{\int}}
\newcommand{\uint}{\overline{\int}}
\newcommand{\hfi}{\hat{f}^{-1}}
\newcommand{\tfi}{\tilde{f}^{-1}}
\newcommand{\tsi}{\tilde{f}^{-1}}
\newcommand{\PP}{\mathcal{P}}
\newcommand{\nin}{\centernot\in}
\newcommand{\seq}[1]{({#1}_n)_{n\geq 1}}
\newcommand{\Tt}{\mathcal{T}}
\newcommand{\card}{\mathrm{card}}
\newcommand{\setc}[2]{\{ #1\::\:#2 \}}
\newcommand{\Fcal}{\mathcal{F}}
\newcommand{\cbal}{\overline{B}}
\newcommand{\Ccal}{\mathcal{C}}
\newcommand{\Dcal}{\mathcal{D}}
\newcommand{\cl}{\overline}
\newcommand{\id}{\mathrm{id}}
\newcommand{\intr}{\mathrm{int}}
\renewcommand{\hom}{\mathrm{Hom}}
\newcommand{\vect}{\mathrm{Vect}}
\newcommand{\Top}{\mathrm{Top}}
\renewcommand{\top}{\Top}
\newcommand{\hTop}{\mathrm{hTop}}
\newcommand{\set}{\mathrm{Set}}
\newcommand{\frp}{\mathop{\large {\mathlarger{*}}}}
\newcommand{\ondt}{1_{\cdot}}
\newcommand{\onst}{1_{\star}}
\newcommand{\bdy}{\partial}
\newcommand{\im}{\mathrm{im}}
\newcommand{\re}{\mathrm{re}}
\newcommand{\oX}{\overline{X}}
\newcommand{\ox}{\overline{x}}
\newcommand{\tX}{\tilde{X}}
\newcommand{\tH}{\tilde{H}}
\newcommand{\tx}{\tilde{x}}
\newcommand{\hX}{\hat{X}}
\newcommand{\hx}{\hat{x}}
\newcommand{\aut}{\mathrm{Aut}}
\newcommand{\del}{\partial}
\newcommand{\RP}{{\RR\mathrm{P}}}
\newcommand{\CP}{{\CC\mathrm{P}}}
\newcommand{\csm}{\RP^n\#\RP^n}
\DeclareMathOperator{\coker}{coker}
\newcommand{\idl}[1]{\langle #1\rangle}
\renewcommand{\thetheorem}{\arabic{section}.\Alph{theorem}}
\DeclareMathOperator{\ext}{Ext}
\newcommand{\tf}{\tilde f}
\DeclareMathOperator{\gl}{GL}
\DeclareMathOperator{\spn}{Span}
\newcommand{\hdr}{H_{\mathrm{dR}}}
\newcommand{\tosm}{\xrightarrow{\sim}}
\newcommand{\oab}{\omega_{\alpha\beta}}
\makeatletter
\newcommand{\extp}{\@ifnextchar^\@extp{\@extp^{\,}}}
\def\@extp^#1{\mathop{\bigwedge\nolimits^{\!#1}}}
\makeatother
\begin{document}
\maketitle
\prob{1} We let $f:S^2\to T^2$ be any smooth map.
\prt{1}
We recall that by de Rahm's theorem, we have an isomorphism of rings $\Psi:\hdr^*(T^2)\tosm H^*(T^2,\RR)$. As we showed in class last semester (or can be found in Hatcher), $H^*(T^2, \RR)$  is generated in degree one by $[\alpha],[\beta]\in H^1(T^2)$ under the relations $\idl{[\alpha]\cup[\beta]+[\beta]\cup[\alpha],[\alpha]^2,[\beta]^2}$. In particular, we have that $H^2(T^2,\RR)\cong \RR\{[\alpha]\cup [\beta]\}$. We let $[\omega_\alpha]=\Psi^{-1}([\alpha])$, and $[\omega_\beta]=\Psi^{-1}([\beta])$. We recall as well that $H^*(S^2,\RR)$ is generated in degree two by a single element $[\gamma]\in H^2(S^2,\RR)$ modulo the relation $\idl{[\gamma]\cup [\gamma]}$. We let $[\omega_\gamma]\in \hdr^2(S^2)$ be its image under de Rahm's theorem. We finally also recall that thanks to de Rahm's isomorphism of rings, the map $f^*$ induces a graded morphism of $\RR$-algebras.
\begin{proposition*}
	For any closed $\omega\in \Omega^2(T^2)$, $f^*(\omega)\in \Omega^2(S^2)$ is exact.
\end{proposition*}
\begin{proof}
We let $\omega\in \Omega^2(T^2)$ be an arbitrary closed $2$-form on $T^2$. We suppose for the sake of contradiction that $f^*(\omega)$ is not exact, that is $f^*([\omega])=r[\omega_\gamma]$ for some $0\neq r\in \RR$. Then, we have two cases to consider. If $\omega$ is exact, we have that $[\omega]=[0]$ in $H^2(T^2)$. Then, $f^*([0])\neq [0]$ contradicting that $f^*$ is a ring homomorphism. On the other hand, if $\omega$ is not exact, then $[\omega]=q[\omega_\alpha]\wedge [\omega_\beta]$ for some $0\neq q \in \RR$. Then, we have that $f^*(q[\omega_\alpha])\wedge f^*([\omega_\beta])=r[\omega_\gamma]$, but as $\hdr^1(S^2)=0$ and $f^*$ is a \emph{graded} homomorphism, we then have that $[0]\wedge [0]=r[\omega_\gamma]$, contradicting that $r\neq 0$. This completes our proof.
\end{proof}
\prt{2}
\begin{proposition*}
	$\deg f=0$. 
\end{proposition*}
\begin{proof}
	As a corollary of de Rahm's theorem that we showed in class, for smooth path-connected $q$-manifolds $M,N$ with smooth $f:M\to N$, $\deg f$ can be calculated as the determinant of $f^*:\hdr^q(N)\to \hdr^q(M)$. As we showed above, $f^*\hdr^2(T^2)\to \hdr^2(S^2)$ is the zero map, and hence $\deg f=0$. 
\end{proof}
\prob{2}
We let $f:\CC^2\to \CC$ be defined by $f(z,w)=w^2-z^3$, and let $X\subseteq \CC^2$ be defined by $X:=\{(z,w)\mid f(w,z)=c\}$.
\prt{1}
We use the diffeomorphism $\CC^2\tosm \RR^4$ by $(r+is,x+iy)\mapsto (r,s,x,y)$ and identify $\CC^2$ as $\RR^4$ as such (and similarly identify $\CC$ with $\RR^2$). Then, $f:\RR^4\to \RR^2$ can be given new coordinates as \[f(r,s,x,y)=(x^2-y^2-r^3+3rs^2,2xy-3r^2s+s^3).\] Then, letting $c=a+ib$, $X$ is defined as $X=X_c:=\{(r,s,x,y)\mid f(r,s,x,y)=(a,b)\}$
\begin{proposition*}
	For any $(a,b)\neq (0,0)$, $X_c$ is a smooth manifold.
\end{proposition*}
 As $f$ is a smooth map with Euclidean domain and codomain, we may compute $\d f_{P}$ as the Jacobian of $f$ at $P=(r,s,x,y)$. Doing so yields the following:
\[\d f_P=\begin{bmatrix}
3(s^2-r^2)&6rs&2x&-2y\\
-6rs&3(s^2-r^2)&2y&2x
\end{bmatrix}\]  
We have that $c=(a,b)\in \RR^2$ is a critical value if for some $P=(r,s,x,y)$ with $f(P)=c$, the derivative $\d f_P$ fails to be surjective, i.e. its maximal minors are identically zero. In particular, if $P$ is a critical point, this forces the relations $\Delta_{12}(\d f_P)=9(s^2-r^2)^2+36r^2s^2=9(s^2+r^2)^2=0$ and $\Delta_{34}(\d f_P)=4(x^2+y^2)=0$. As $r,s,x,y$ are \textbf{real} coordinates, these two relations force $(r,s,x,y)=\avec{0}$. Hence, $\avec{0}\in \RR^4$ is the only critical point of $f$, and $\avec{0}\in \RR^2$ its corresponding critical value. By the regular value theorem, for $c=(a,b)\neq (0,0)$, we have that $X_c$ is a smooth manifold of dimension $\dim \RR^4-\dim \RR^2=2$.
\prt{2}
We let\footnote{Oops, I'm changing names on you\textemdash my bad.} $g:\CC^2\to \CC$ be defined by $g(z,w)=w$. Under our change of coordinates above, this is the map $g:\RR^4\to \RR^2$ defined by $g(r,s,x,y)=(x,y)$. 
\begin{proposition*}
	The critical points of $g$ are exactly the plane $r=s=0$. 
\end{proposition*}
\begin{proof}
	We have by a proposition on page 24 of Guillemin-Pollack that $T_PX=\ker \d f_P$. Assuming that $P\neq \avec{0}$ (justified by part a), we compute manually that 
	\[\ker \d f_P\supseteq \spn\left\{v_1=\begin{bmatrix}
	2x\\2y\\-3(s^2-r^2)\\6rs
	\end{bmatrix},v_2=\begin{bmatrix}
	2y\\-2x\\6rs\\3(s^2-r^2)
	\end{bmatrix}\right\}\]
	We note that the linear independence of $v_1$ and $v_2$ can be checked by computing the rank of the matrix $\begin{bmatrix} v_1&v_2\end{bmatrix}$. Doing so reveals the same maximal minors as $\d f_P$, implying that $v_1$ and $v_2$ span the tangent space at $P$ for any $P\in X$ by part a. We take $v_1$ and $v_2$ to be a basis for $T_PX$. Then, the inclusion map $i:T_PX\to T_P\RR^4$ is given by $i(v)=\begin{bmatrix}
	v_1&v_2
	\end{bmatrix}v$. We let $h$ be the extension of $g$ to all of $\RR^4$ given coordinate-wise by the same formula as $g$. Then, we may compute $\d h_P$ as \[\d h_P=\begin{bmatrix}
	0&0&1&0\\0&0&0&1
	\end{bmatrix}. \] Furthermore, we have that $g=h\circ i$ and thus $\d g_P=\d h_{i(P)}\d i_P$. We compute \[\d g_P=(\d h_{i(P)}\circ\d i_P)(v)=\begin{bmatrix}
	0&0&1&0\\0&0&0&1
	\end{bmatrix}\begin{bmatrix}
	2x&2y\\2y&-2x\\-3(s^2-r^2)&6rs\\6rs&3(s^2-r^2)
	\end{bmatrix}v=\begin{bmatrix}
	-3(s^2-r^2)&6rs\\6rs&3(s^2-r^2)
	\end{bmatrix}v \]
	We have that $P$ is a critical point of $g$ if $\d g_P$ fails to be surjective, i.e. fails to be an isomorphism. We compute $\det \d g_P=9(s^2+r^2)^2$ and have that $\det \d g_P=0\iff (s,r)=(0,0)$. As $\det \d g_P=0$ if and only if $P$ is a critical point, this proves our claim.
\end{proof}
\prob{3}
\begin{proposition*}
	If $\omega$ is a closed 2-form on $S^4$, then $\omega\wedge\omega$ vanishes somewhere on $S^4$.
\end{proposition*}
\begin{proof}
We break our proof into a series of claims.\begin{claim}
	$\omega \wedge \omega$ is closed
\end{claim}
\begin{subproof}
We write $d(\omega\wedge \omega)=d\omega\wedge \omega+\omega\wedge d\omega=0\wedge\omega +\omega \wedge 0=0.$
\end{subproof}
\begin{claim}
$\omega \wedge \omega$ is exact.
\end{claim}
\begin{subproof}
	We recall that de Rahm's isomorphism $\Psi:\hdr^*(S^4)\tosm H^*(S^4,\RR)$ induces a ring structure via the wedge product on $\hdr^*(S^4)$. We have that $\omega$ is necessarily exact as it is closed and $H^2(S^4)=0$. Thus, $[\omega]=[0]\in H^2(S^4)$, so if $[\omega\wedge\omega]\neq [0]$, we have that $0\neq[\omega\wedge\omega]= [\omega]\wedge [\omega]=[0]\wedge [0]=[0]$, a contradiction. Thus, $[\omega\wedge\omega]=[0]$ indeed, i.e. $\omega\wedge\omega$ is exact.
\end{subproof}
As an immediate corollary, as de Rahm's isomorphism works by integration on $\hdr^4(S^4)$, we have that $\int_{S^4}\omega \wedge \omega=0$.
\begin{lemma}[Mean value theorem for smooth manfiolds]
	If $M$ is a $d$-manifold and $\sigma\in\Omega^d(M)$ is exact, then there is some point $p\in M$ such that where $\omega=f(x)\d x_1\wedge \dots \wedge\d x_d$, $f(p)=0$. 
\end{lemma}
\begin{subproof}
\todo{use the mean value theorem for integration}
\end{subproof}
\end{proof}
\prob{4}
\prt{1}
\prt{2}
\prob{5}

\end{document}